\section{Introduction}
\label{sec:introduction}

%For instance, a typical user may view emails on his/her Android or iOS or WindowsPhone handheld
%device, such as a mobile phone or a tablet. 
%The user then can switch to a desktop environment, such as Windows or Linux or OSX 
%in an workplace like setting through either a web browser or 
%a standalone application, such as Microsoft Outlook or Mozilla Thunderbird.

Developers are increasingly releasing different versions of their applications to attract new users and to retain existing users across different platforms.
For example, a typical mobile software developer often releases his/her applications
on all the popular mobile platforms, such as Android, iOS, and Windows,
which often involves rewriting applications in different languages.
For instance, Java is preferred language for implementing Android applications
and Objective-C for iOS application.
In context of desktop software, many well-known projects, such as JUnit and
Hibernate provide multiple versions in different programming languages,
to attract the developer community to use these
libraries across those languages.


%However, manually migrating a software from one platform (/language) to another
%can be time consuming and may be error prone.

Existing tools such as Java2CSTranslator~\cite{java2cstranslator}
assist developers by automating the process of software migration.
However, such tools require a programmer to manually input
how methods in a source language's Application Programming Interfaces (API) maps to the methods of the target language's API. 
Given a typical language (or platform) exposes a large number of API methods for developers to reuse, manually writing these mappings is prohibitively resource intensive and may be error prone.


Existing program-analysis based approaches
address the problem of finding method mapping between APIs using
static~\cite{Zhong2010ICSE} and dynamic~\cite{Gokhale2013ICSE} analysis. 
Recently Nguyen et al.~\cite{nguyen2014statistical} further proposed to
apply statistical language translation techniques to achieve language migration
by mining large corpora of open source software repositories.
However, these approaches require as an input manually ported software
or at least functionally similar software across source and target APIs.
Since static analysis and mining approaches~\cite{Zhong2010ICSE,nguyen2014statistical}
leverage source code analysis, 
accuracy of such approaches is dependent on the quality of the code under consideration.
Likewise, accuracy of dynamic approaches~\cite{Gokhale2013ICSE} is dependent on
the quality and completeness of test inputs
to dynamically execute the API behavior comprehensively. 
Furthermore, such program-analysis approaches are often sensitive to the nuances of
the programming language of the source-code under analysis, thus require significant effort
for accommodating new programming languages.  


\textit{The goal of this research is to support software developers 
	in migrating an application from a source API to a target API
	by automatically discovering relevant method mappings across APIs using text mining
	on the natural language API method descriptions.}

We propose to use the natural language API method descriptions
to discover the method mappings across APIs.
We hypothesize:
\textit{since the API documents are targeted towards developers,
	there may be an overlap in the language used to describe similar concepts that can be leveraged.}
In general, API documentation provides developers with useful information
about class/interface hierarchies within the software.
Additionally, API documents also provide information about
how to use a particular method within a class by means of method descriptions.
A method description typically outlines specifications in terms of
the expectations of the method arguments and functionality of method in general.


This paper presents \tool : An approach that leverages the natural language method descriptions to discover the likely method mapping between APIs.
\tool\ stands for \textit{\textbf{T}ext \textbf{M}ining
	based approach to discover likely \textbf{AP}I method mappings}.
\tool\ accepts as input the the API doucments of source and target API.
In particular, \tool\ proposes to create a vector space model~\cite{singhal2001modern,manning2008introduction} of the target API method descriptions. 
\tool\ then queries the vector space model of target API using
automatically generated queries from the source API method descriptions.
\tool\ automates the query generation in source API using the concepts from text mining, such as emphasizing and de-emphasizing certain keywords over others and querying multiple facets, such as class description, package names, and method description.
The output of \tool\ is a ranked list of methods of the target API that are candidates for the mapping of the method from the source API that was used to generate query.
Since \tool\ analyzes API documents in natural language, the proposed approach is reusable, independent of the programming language of the library.


However, its challenging to automatically generate the queries. In particular, given a typical API exposes a large number of methods, a large search space may result noise in the search results.
Consider for example, Android API level 23 exposes 4404~\footnote{\url{http://developer.android.com/reference/classes.html}} classes
and Java 8 exposes 4240~\footnote{\url{https://docs.oracle.com/javase/8/docs/api/allclasses-noframe.html}} classes.
Furthermore, each class exposes several methods resulting in large search space for each query resulting in a low effectiveness of search. 
\tool\ addresses this challenge by leveraging user feedback in the form of confirmed mappings to prune the search space using topic modelling~\cite{blei2003latent,panichella2013effectively}. 


We pose the following research questions:
\begin{itemize}
	
	\item\textbf{RQ1}: Does \tool\ discover additional method mappings in comparison to existing program-analysis based approaches?
	
	\item\textbf{RQ2}: What is the improvement in the rankings of the method mappings in \tool\ output by leveraging the use feedback?
	
	\item\textbf{RQ3}: What is the overlap of the method mappings discovered by \tool\ in comparison with the mappings discovered by existing program-analysis based approaches?
	
	
	\item\textbf{RQ4}: Does \tool\ discover relevant methods using free-form queries instead of automated generated queries?
	
\end{itemize}
To answer our questions, we apply \tool\ to discover likely API mappings for 15 classes across:
1) \CodeIn{Java} and \CodeIn{C\#} API; 2) \CodeIn{Java ME}~\footnote{Java Platform Micro Edition} and \CodeIn{Android} API.
We also compare the discovered mappings with two state-of-the-art program-analysis based approaches: Rosetta~\cite{Gokhale2013ICSE} and StaMiner~\cite{nguyen2014statistical}.
%Our results indicate that \tool\ on an average found relevant mappings for 57\% more methods compared to Rosetta and StaMiner. 
%Furthermore, our results also indicate that \tool\ found on average exact mappings for 6.5 more methods per class with a maximum of 21 additional exact mappings for a single class as compared to previous approaches.

%In summary, \tool\ leverages natural language description of APIs to discover likely mapping thus facilitating cross API migration of applications. Since \tool\ analyzes API documents in natural language, the proposed approach is reusable, independent of the programming language of the library. 

This paper makes the following major contributions:
\begin{itemize}
	\item A text mining (queries on vector space model) based approach that effectively discovers mapping between source and target API.
	\item A topic modeling based approach to prune the search space of queries by using end user feedback to improve the effectiveness of discovering mapping between source and target API. 
	\item A prototype implementation of our approach based on extending the Apache Lucene~\cite{lucene} and Mallet~\cite{McCallumMALLET}. An open source implementation of our prototype can be found at our website~\cite{projectWeb}. 
	\item An evaluation of our approach on 5 classes in Java ME to Android API and 10 Classes Java to C\# API. The evaluation results and artifacts are publicly available on the project website.
\end{itemize}


The rest of the paper is organized as follows.
Section~\ref{sec:background} presents the background on Text Mining.
Section~\ref{sec:example} presents a real world example that motivates our approach. Section~\ref{sec:approach} presents \tool\ approach.
Section~\ref{sec:evaluation} presents evaluation of \tool.
Section~\ref{sec:discussion} presents a brief discussion and future work.
Section~\ref{sec:related} discusses the related work.
Finally, Section~\ref{sec:conclusion} concludes the paper.


\begin{framed}
This paper is a revised, expanded version of a paper (Pandita et al.~\cite{pandita2015discovering}) presented at the 15th
IEEE working conference on Source code analysis and Maintenance (SCAM 2015).
\end{framed}


