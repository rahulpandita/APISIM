\section{Limitation and Future work}
\label{sec:discussion}

We next describe the limitations and future work of \tool,
followed by discussions on threats to validity.


A key limitation of the presented work is its reliance on the human developer
to confirm or refute mappings.
In the future, we plan to extend the \tool\ infrastructure to achieve an end-to-end automation. 
Particularly, we plan on using the program-analysis techniques, 
such as type-analysis proposed in
existing approaches~\cite{nguyen2014statistical,zhong09SE}.



Sometimes the functionality achieved by a method call in a source API,
is achieved by a sequence of method calls in the destination API and vice versa.
Although \tool\ may return individual methods as relevant, 
\tool\ does not provide explicit sequences of method calls as relevant suggestions.
In the future, we plan to extend the current text mining infrastructure
to provide method sequences as relevant suggestions when applicable. 
In particular, we plan to leverage the NLP techniques, such as
specification inference~\cite{pandita12:inferring} to identify method sequences. 



From an implementation perspective, \tool\ does not take into account 
the API fields,
which limits \tool s ability in reporting mappings involving API fields.
%For instance, the functionality achieved by a method call in a source API
%may be achieved by accessing a API field in destination API.
However, disregarding API fields is a limitation of the current implementation 
and in future iterations of \tool\ implementations we plan to include API fields in the indexes as well.

Finally, \tool\ operates under the assumption of the availability of the API documents.
Thus \tool\ is not applicable in situations where API documents are of low quality or are unavailable altogether.
In the future, we plan to extend \tool\ infrastructure to workaround such situations
by integrating with existing source-code-mining based approaches.
Specifically we plan to leverage techniques like code summarisation~\cite{sridhara2011ICPC} and IR based approaches like Exoa~\cite{kim2010towards}.
%We do not guarantee the exhaustive mapping

%limitations with respect to most commonly occurring method names like \CodeIn{get} \CodeIn{set}


 

\textbf{Threats to Validity}: The primary threat to external validity is the representativeness of our experimental subjects to the real world software.
To address this threat we chose real world API pairs:
1) Java ME and Android APIs are two Java based platforms to develop mobile applications; 
2) Java and C\# APIs are the top object-oriented programming language APIs used for generic application development. 
The threat can be further minimized by evaluating \tool\ on more APIs from different domains. 

Our chief threat to internal validity is the accuracy of \tool\ in identifying API mappings. 
To minimize this threat we compared the mappings inferred by \tool\ with
the mappings provided by previous approaches. We thank Gokhale et al.~\cite{Gokhale2013ICSE} for sharing with us the Java ME and Android API mappings inferred by their Rosetta approach. We also thank Nguyen et al.~\cite{nguyen2014statistical} for making their mappings publicly available. Furthermore, authors did manually identify some of the mappings that could not be compared to previous work.
Thus, human errors may affect our results. 
To minimize the effect, each annotation was independently agreed upon by two authors.