\section{Conclusion}
\label{sec:conclusion}


%API documents provide specification information about how to use a particular method within a class by means of method descriptions. However, specifications described in natural Language in API documents are not amenable to formal verification by existing verification tools. 

%Specifications described in natural language in API documents are not amenable to formal verification by existing verification tools. In this paper, we have presented a novel approach for inferring formal specifications from API documents targeted towards code contract generation. Our evaluation results show that our approach has an average of 92\% precision and 93\% recall in identifying sentences describing code contracts from over 2500 sentences. Furthermore, our results also show that our approach has an average of 83.4\% accuracy in inferring specifications from sentences describing code contracts out of over 1600 sentences. 

API mapping across different platforms/languages mappings facilitate machine-based migration
of an application from one API to another.
Thus tool assisted discovery of such mappings is highly desirable.
In this paper, we presented \tool : a lightweight text-mining based approach
to infer API mappings.
\tool\ compliments existing mapping inference techniques by leveraging natural language descriptions in API documents instead of relying on source code.
We used \tool\ approach to discover API mappings for 15 types across: 
1) \CodeIn{Java} and \CodeIn{C\#} API,
2) \CodeIn{Java ME} and \CodeIn{Android} API.
We demonstrated the effectiveness of \tool\ by 
comparing the discovered mappings with state-of-the-art code analysis based approaches.
Our results indicate that \tool\ on average found relevant mappings for 57\% more methods compared to previous approaches. 
Furthermore, our results also indicate that \tool\ found on average exact mappings for 6.5 more methods per type with a maximum of 21 additional exact mappings for a single type as compared to previous approaches.