\section{Evaluation}
\label{sec:evaluation}

We conducted an evaluation to assess the effectiveness of our approach. In our evaluation, we address three main research questions:

\begin{itemize}
	
\item\textbf{RQ1}: What are the precision and recall of our approach in identifying API mappings?

\item\textbf{RQ2}: How do the mappings inferred by our approach compare with the human written mappings?

\item\textbf{RQ3}: What is the effectiveness of using free form queries using our approach?

\end{itemize}

\subsection{Subjects}
\label{sub:subject}

We used the API documents of the following library pairs as subjects for our evaluation. 


\textbf{J2ME-Android}. Java Platform, Micro Edition, or Java ME, is a Java platform designed for embedded systems (mobile devices are one kind of such systems). Target devices range from industrial controls to mobile phones (especially feature phones) and set-top boxes. Java ME was formerly known as Java 2 Platform, Micro Edition (J2ME).

Android is a Linux-based operating system designed primarily for touchscreen mobile devices such as smartphones and tablet computers.

\textbf{Java -C\#}




Effectiveness in free form queries:

\begin{table}
	\begin{center}	
		\caption{Comparison with SNIFF}
			\begin{tabular}{lrr}
				\topline
				\headcol Query	& \multicolumn{2}{c}{Relevant Mtd rank}\\
				\headcol 		& SNIFF	& \tool \\
				\midline 
				
				\rowcol get active editor window from eclipse workbench	& 1 & 1\\
				\rowpln parse a java source and create ast & 1 & 1\\
				\rowcol connect to a database using jdbc & 1 & 1\\
				\rowpln display directory dialog from viewer in eclipse & 1 & 1\\
				\rowcol read a line of text from a file & 1 & 1\\
				\rowpln return an audio clip from url & 1 & 1\\
				\rowcol execute SQL query & 2 & 1\\
				\rowpln current selection from eclipse workbench & 1 & 1\\ 
				\bottomline
				%----------------- END TABLE DATA ------------------------ 
			\end{tabular}
			\label{tab:SNIFFComp}
	\end{center}
\end{table}


\begin{table}
	\begin{center}	
		\caption{Comparison with Rosetta}
		\begin{tabular}{rlr}
			\topline
			\headcol S. No.	& Method & \tool\ Rank\\
			\midline

				\rowcol	1	&	Alert.setTimeout	&	\\
				\rowpln	2	&	Alert.setType	&	\\
				\rowcol	3	&	Canvas.getHeight	&	\\
				\rowpln	4	&	Canvas.getWidth	&	\\
				\rowcol	5	&	Canvas.repaint	&	\\
				\rowpln	6	&	Canvas.serviceRepaints	&	\\
				\rowcol	7	&	Canvas.setFullScreenMode	&	\\
				\rowpln	8	&	Command.getCommandType	&	\\
				\rowcol	9	&	Command.getLabel	&	\\
				\rowpln	10	&	Display.getCurrent	&	\\
				\rowcol	11	&	Display.getDisplay	&	\\
				\rowpln	12	&	Display.setCurrent	&	\\
				\rowcol	13	&	Displayable.addCommand	&	\\
				\rowpln	14	&	Displayable.getHeight	&	\\
				\rowcol	15	&	Displayable.getWidth	&	\\
				\rowpln	16	&	Displayable.removeCommand	&	\\
				\rowcol	17	&	Font.charWidth	&	\\
				\rowpln	18	&	Font.getHeight	&	\\
				\rowcol	19	&	Font.stringWidth	&	\\
				\rowpln	20	&	Form.addCommand	&	\\
				\rowcol	21	&	Form.setCommandListener	&	\\
				\rowpln	22	&	game.GameCanvas.getHeight	&	\\
				\rowcol	23	&	game.GameCanvas.getWidth	&	\\
				\rowpln	24	&	game.GameCanvas.repaint	&	\\
				\rowcol	25	&	game.GameCanvas.serviceRepaints	&	\\
				\rowpln	26	&	game.GameCanvas.setFullScreenMode	&	\\
				\rowcol	27	&	game.Layer.getHeight	&	\\
				\rowpln	28	&	game.Layer.getWidth	&	\\
				\rowcol	29	&	game.Layer.setPostion	&	\\
				\rowpln	30	&	game.Sprite.paint	&	\\
				\rowcol	31	&	game.Sprite.setFrame	&	\\
				\rowpln	32	&	game.Sprite.setPosition	&	\\
				\rowcol	33	&	game.Sprite.setRefPixelPosition	&	\\
				\rowpln	34	&	Graphics.clipRect	&	\\
				\rowcol	35	&	Graphics.drawArc	&	\\
				\rowpln	36	&	Graphics.drawChar	&	\\
				\rowcol	37	&	Graphics.drawImage	&	\\
				\rowpln	38	&	Graphics.drawLine	&	\\
				\rowcol	39	&	Graphics.drawRect	&	\\
				\rowpln	40	&	Graphics.drawString	&	\\
				\rowcol	41	&	Graphics.fillArc	&	\\
				\rowpln	42	&	Graphics.fillRect	&	\\
				\rowcol	43	&	Graphics.fillRoundRect	&	\\
				\rowpln	44	&	Graphics.fillTriangle	&	\\
				\rowcol	45	&	Graphics.getClipHeight	&	\\
				\rowpln	46	&	Graphics.getClipWidth	&	\\
				\rowcol	47	&	Graphics.getClipX	&	\\
				\rowpln	48	&	Graphics.getClipY	&	\\
				\rowcol	49	&	Graphics.setClip	&	\\
				\rowpln	50	&	Graphics.setColor	&	\\
				\rowcol	51	&	Graphics.setFont	&	\\
				\rowpln	52	&	Image.createImage	&	\\
				\rowcol	53	&	Image.getGraphics	&	\\
				\rowpln	54	&	Image.getHeight	&	\\
				\rowcol	55	&	Image.getWidth	&	\\
				\bottomline
				%----------------- END TABLE DATA ------------------------ 
		\end{tabular}
		\label{tab:RosettaComp}
	\end{center}
\end{table}

Discussion with Authors

None of the mapping methods point to AlertDIalog Class in Android that seems to be the class implementing alert functionality.
Furthermore, consider the first-mapping

\CodeIn{Alert.setString Paint.setAlpha; CompundButton.setChecked}

None of the mapped methods set text in the alert. How to interpret the mapping in such cases?

Amruta's Response :


First, a note of clarification. As explained in the paper, we restricted ourselves to sequences of length up to 2 when inferring mappings (i.e., {A->p;q}, or {A->p}). So this might mean that p-q only implements some of the functionality of A on the target platform, and we still count this as a valid mapping.

With that said, let me answer your questions: Let's consider the mapping: "Alert.setString -> Paint.setAlpha; CompundButton.setChecked". In many of our traces, we observed that setting a string first involves setting the attributes of the paint (with is used to draw the text), followed by a call to setText method, which led us to believe that the sequence "Paint.setAlpha; CompundButton.setChecked" could be a likely mapping, if at least in part. The same reasoning applies to mappings of the other two methods in your list. 

Technically, you're right that "Alert" related methods would be implemented using methods of the "AlertDialog" class. However, "Alert" is used to display messages, and we considered Android methods/sequences that could also be used to display messages as valid mappings. These methods/sequences may not strictly be considered "AlertDialogs" in the Android sense, but can still be part of rendering a text message on the screen. 

\begin{table*}
	\begin{center}	
		\caption{Comparison with Tien Nyugen}
		\begin{tabular}{rllr}
			\topline
			\headcol S.No.	&	Source Class	&	Target Class	&	\tool\ Ranking \\
			\midline

				\rowcol	
				\rowpln	1	&	java.util.concurrent.atomic.AtomicBoolean	&		&	\\
				\rowcol	2	&	java.util.jar.JarInputStream	&		&	\\
				\rowpln	3	&	java.util.jar.Attributes	&		&	\\
				\rowcol	4	&	java.util.jar.Manifest	&		&	\\
				\rowpln	5	&	java.util.LinkedHashMap	&	System.Collections.Hashtable	&	\\
				\rowcol	6	&	java.util.AbstractCollection	&	System.Collections.ICollection	&	\\
				\rowpln	7	&	java.util.Date	&	System.DateTime	&	\\
				\rowcol	8	&	java.util.HashSet	&	ILOG.J2CsMapping.Collections.HashedSet	&	\\
				\rowpln	9	&	java.util.SortedSet	&	ILOG.J2CsMapping.Collections.SortedSet	&	\\
				\rowcol	10	&	java.util.Set	&	ILOG.J2CsMapping.Collections.ISet	&	\\
				\rowpln	11	&	java.util.TreeSet	&	ILOG.J2CsMapping.Collections.SortedSet	&	\\
				\rowcol	12	&	java.util.BitSet	&	ILOG.J2CsMapping.Collections.BitSet	&	\\
				\rowpln	13	&	java.util.Hashtable	&	System.Collections.Hashtable	&	\\
				\rowcol	14	&	java.util.TreeMap	&	System.Collections.SortedList	&	\\
				\rowpln	15	&	java.util.SortedMap	&	System.Collections.SortedList	&	\\
				\rowcol	16	&	java.util.Iterator	&	ILOG.J2CsMapping.Collections.IIterator	&	\\
				\rowpln	17	&	java.util.IdentityHashMap	&	System.Collections.Hashtable	&	\\
				\rowcol	18	&	java.util.Vector	&	System.Collections.ArrayList	&	\\
				\rowpln	19	&	java.util.WeakHashMap	&	System.Collections.Hashtable	&	\\
				\rowcol	20	&	java.util.StringTokenizer	&	ILOG.J2CsMapping.Util.StringTokenizer	&	\\
				\rowpln	21	&	java.util.EventObject	&	ILOG.J2CsMapping.Util.EventObject	&	\\
				\rowcol	22	&	java.util.Locale	&	System.Globalization.CultureInfo	&	\\
				\rowpln	23	&	java.util.Map.Entry	&	System.Collections.DictionaryEntry	&	\\
				\rowcol	24	&	java.util.UUID	&	System.Guid	&	\\
				\rowpln	25	&	java.util.EventListener	&	ILOG.J2CsMapping.Util.IEventListener	&	\\
				\rowcol	26	&	java.util.AbstractList	&	System.Collections.ArrayList	&	\\
				\rowpln	27	&	java.util.TimerTask	&	ILOG.J2CsMapping.Util.IlTimerTask	&	\\
				\rowcol	28	&	java.util.Timer	&	ILOG.J2CsMapping.Util.IlTimer	&	\\
				\rowpln	29	&	java.util.EmptyStackException	&	System.Exception	&	\\
				\rowcol	30	&	java.util.Collections	&	ILOG.J2CsMapping.Collections.Collections	&	\\
				\rowpln	31	&	java.util.ListIterator	&	ILOG.J2CsMapping.Collections.IListIterator	&	\\
				\rowcol	32	&	java.util.HashMap	&	System.Collections.Hashtable	&	\\
				\rowpln	33	&	java.util.Calendar	&	ILOG.J2CsMapping.Util.Calendar	&	\\
				\rowcol	34	&	java.util.ResourceBundle	&	System.Resources.ResourceManager	&	\\
				\rowpln	35	&	java.util.Properties	&	ILOG.J2CsMapping.Util.Properties	&	\\
				\rowcol	36	&	java.util.Stack	&	System.Collections.Stack	&	\\
				\rowpln	37	&	java.util.MissingResourceException	&	System.Resources.MissingManifestResourceException	&	\\
				\rowcol	38	&	java.util.Comparator	&	System.Collections.IComparer	&	\\
				\rowpln	39	&	java.util.ArrayList	&	System.Collections.ArrayList	&	\\
				\rowcol	40	&	java.util.Collection	&	System.Collections.ICollection	&	\\
				\rowpln	41	&	java.util.Map	&	System.Collections.IDictionary	&	\\
				\rowcol	42	&	java.util.ConcurrentModificationException	&	System.InvalidOperationException	&	\\
				\rowpln	43	&	java.util.Arrays	&	System.Array	&	\\
				\rowcol	44	&	java.util.Currency	&		&	\\
				\rowpln	45	&	java.util.NoSuchElementException	&	System.InvalidOperationException	&	\\
				\rowcol	46	&	java.util.Enumeration	&	ILOG.J2CsMapping.Collections.IIterator	&	\\
				\rowpln	47	&	java.util.LinkedList	&	ILOG.J2CsMapping.Collections.LinkedList	&	\\
				\rowcol	48	&	java.util.GregorianCalendar	&	ILOG.J2CsMapping.Util.GregorianCalendar	&	\\
				\rowpln	49	&	java.util.Random	&	System.Random	&	\\
				\rowcol	50	&	java.util.PropertyResourceBundle	&		&	\\
				\rowpln	51	&	java.util.List	&	System.Collections.IList	&	\\
				\rowcol	52	&	java.util.regex.Pattern	&	ILOG.J2CsMapping.Text.Pattern	&	\\
				\rowpln	53	&	java.util.regex.Matcher	&	ILOG.J2CsMapping.Text.Matcher	&	\\
				\bottomline
				%----------------- END TABLE DATA ------------------------ 
			\end{tabular}
			\label{tab:J2CSharpComp}
		\end{center}
	\end{table*}