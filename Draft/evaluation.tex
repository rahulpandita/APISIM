\section{Evaluation}
\label{sec:evaluation}

We conducted an evaluation to assess the effectiveness of \tool. In our evaluation, we address following research questions:

\begin{itemize}
	
\item\textbf{RQ1}: What is the effectiveness of \tool\ in leveraging the similarity in the language of API method descriptions to discover likely API Mappings?

\item\textbf{RQ2}: How do the mappings discovered by \tool\ compare with the human written mappings?

%\item\textbf{RQ3}: What is the effectiveness of using free form queries using \tool?

\end{itemize}

\subsection{Subjects}
\label{sub:subject}


We evaluated \tool\ using a snapshot of the publicly available API documents of Java, C\#, Android, and Java ME downloaded on Jan 2015. 

Java Platform Micro Edition, or Java ME, is a Java platform designed for embedded systems (such as mobile devices). Target devices range from industrial controls to mobile phones (especially feature phones) and set-top boxes.
Android is a linux-based operating system designed primarily for touchscreen mobile devices, such as smartphones and tablet computers.
%API documents for both Java ME and Android are publicly available at
%\url{http://docs.oracle.com/javame/config/cldc/ref-impl/midp2.0/jsr118/index.html}
%and 
%\url{http://developer.android.com/reference/packages.html}
%respectively.

Java and C\# are general-purpose programming languages from Oracle and Microsoft respectively. Java applications are typically compiled to bytecode that run on any Java Virtual Machine (JVM) irrespective of underlying computer architecture.
Likewise, C\# is compiled into intermediate representation that run on Microsoft's common language infrastructure.
%API documents for both Java and C\# are publicly available at
%\url{http://docs.oracle.com/javase/8/docs/api/}
%and
%\url{https://msdn.microsoft.com/en-us/library/gg145045(v=vs.110).aspx}
%respectively.


Particularly we used the API documents of the following library pairs as subjects for our evaluation. 


\textbf{Java ME (to Android) API}:
For our evaluation we considered the methods in the following Java ME Types as the source API methods to discover mapping methods in Android API: 
\CodeIn{Alert},
\CodeIn{Canvas},
\CodeIn{Command},
\CodeIn{Graphics}, and
\CodeIn{Font}
Classes in \CodeIn{javax.} \CodeIn{microedition.} \CodeIn{lcdui} package.


The listed types provides methods for supporting graphics related functionality in Java ME.
Furthermore, Gokhale et al.~\cite{Gokhale2013ICSE} approach Rossetta reports the
mapping for methods in theses types along with seven others (twelve types in total) as a part of their evaluation,
thus providing for a baseline comparison with dynamic analysis based approaches.
Rosetta approach requires a user to manually execute functionally similar applications using source and target API with identical (or near identical) inputs and collect execution traces.
Finally Rosetta analyses the collected execution traces to infer method mappings.
We focused our evaluation on the listed five types which first three authors perceived as frequently used types among the twelve types reported by Rosetta.
In the future, we plan to evaluate \tool\ approach on all the twelve reported types. 

\textbf{Java (to C\#) API}:
For our evaluation we considered the methods in the following Java Types as the source API methods to discover mapping methods in C\# API: 
1) \CodeIn{File},
\CodeIn{Reader}, and 
\CodeIn{Writer} in \CodeIn{java.io} package;
2) \CodeIn{Calendar},
\CodeIn{Iterator},
\CodeIn{HashMap}, and 
\CodeIn{ArrayList} in \CodeIn{java.util} package; and
3) \CodeIn{Connection},
\CodeIn{ResultSet}, and 
\CodeIn{Statement} classes in \CodeIn{java.sql} package.
 
Types in \CodeIn{java.io} provide the API methods for accessing and manipulating the file system.
Types in \CodeIn{java.util} provide API methods for miscellaneous utilities, such as text manipulation, collections frameworks and other data structures.
Types in \CodeIn{java.sql} provide the API methods for accessing and processing data stored in databases.
We selected these particular packages in Java programming languages because Nguyen et al.~\cite{nguyen2014statistical} in their work (StaMiner) for statistical language migration find mappings for the types in these packages.
Their mappings provides for a external baseline comparison.
Although, Nguyen et al.~\cite{nguyen2014statistical} report on all the classes in these packages,
due to the amount of effort, we focused our analysis on the listed types which first three authors perceived as frequently used types in their respective packages.


\subsection{Evaluation Setup}


We first downloaded the publicly available API documents from the respective websites of the subject APIs. We then cleaned and extracted the desired fields as described in the Section~\ref{sub:Approach_Indexer}. We then indexed the extracted text into the Lucene indexes.
We created a separate index for every API type: Java ME, Android, Java, and C\#. 


For every class/interface under consideration (as listed in Section~\ref{sub:subject}),
we extracted the publicly listed methods from API documents.
For a given type, we only consider the methods that are explicitly listed in the public API.
We exclude the methods that are inherited from a parent type without modification.
For instance, we exclude the \CodeIn{equals} method from the parent class \CodeIn{Object} which is the parent class for majority of Java types.
Since, the inherited methods are part of the parent type,
\tool\ will find the mapping when considering the parent type.
We then use \tool\ to create the queries form the descriptions of the considered methods as described in Section~\ref{sub:Approach_Searcher}.
Finally, we execute the formulated queries on the index and collect results
We only consider top-10 results for each query.
Previous approaches~\cite{chatterjee2009sniff,Gokhale2013ICSE}
also only consider the top-10 results suggested by their approach for evaluation.


The top-10 matches found by the \tool\ are then analyzed/reviewed manually to determine the effectiveness of the matched results.
For a given method in the source API, a match is characterized by a class and a method within that class that is determined to be the corresponding implementation in the target API.
Authors next annotated each match as `relevant' and/or `exact' based on the following acceptance criteria:
\begin{enumerate}
	\item{\textbf{Relevant}} If the target method in the top-10 list can be used to implement the same (or similar) functionality as the source method, we classify the result as relevant.\\
	OR\\
	The target method is reported by the previous approaches~\cite{Gokhale2013ICSE,nguyen2014statistical} as a mapping.

	\item{\textbf{Exact}} If the target method is a relevant method and the target method accurately captures the functionality of the source method, and implements the same feature/function, the resultant match is classified as an `exact' match.\\
	OR\\
	The target method is reported by the previous approaches~\cite{Gokhale2013ICSE,nguyen2014statistical} as a mapping.
\end{enumerate}

For example, the method \CodeIn{getInt} in the type \CodeIn{java.sql.ResultSet} has an exact match in the C\# method \CodeIn{GetInt32} from \CodeIn{system.data.} \CodeIn{sqlclient.SqlDataReader} type, since it provides the same functionality of extracting the value stored in a specified column as a 32-bit signed integer.
On the other hand, the method \CodeIn{getClob} in type \CodeIn{java.sql.ResultSet} does not have an exact corresponding method in C\#. The closest functionality available is the C\# method \CodeIn{GetValues} in the class \CodeIn{system.data.sqlclient.SqlDataReader}. 
Thus the method \CodeIn{GetValues} is marked as relevant, but not an exact match.


We then calculate coverage ($Cov$) as the ratio of the number of source methods in a type to the number of methods that \tool\ found \textit{at-least} one relevant mapping.
We also calculate the $\Delta_{Cov}$ as increase in the $Cov$ in comparison to results reported by previous approaches~\cite{Gokhale2013ICSE,nguyen2014statistical}. High value of $\Delta_{Cov}$ indicates the effectiveness of \tool\ in finding API method mappings.

Finally we measure the common methods between the exact mappings suggested by \tool\ for a source method with the mappings suggested by previous approaches.
We then calculate, the number of new mappings a the number of exact mappings minus the common mappings.

\subsection {Results}

We next describe our evaluation results to demonstrate the effectiveness of \tool\ in leveraging natural language API descriptions to discover method mappings across APIs.

\subsubsection{RQ1: Effectiveness of \tool\ }

\begin{table*}
	\begin{center}	
		\caption{Evaluation Results}
		\begin{tabular}{rlllr|rr|rr|rrr}
				\topline
				\headcol 		& \multicolumn{2}{c}{API}	&		& No.		& \multicolumn{2}{c|}{Relevant} 	& \multicolumn{2}{c|}{Exact} &					&			&		\\
				\headcol S No.	& Source	& Target		& Type	& Methods	& Prev 	& \tool 				& Prev	& \tool 			& $\Delta_{Cov}$	& {\small Common}	& New	\\
				\midline

				\rowcol	1	& Java ME	& Android	& javax.microedition.lcdui.Alert	& 16	& 3		& 15	& 3		& 7		& 0.75	& 0	& 7		\\
				\rowpln	2	& Java ME	& Android	& javax.microedition.lcdui.Canvas	& 22	& 5		& 18	& 5		& 10	& 0.60	& 0	& 10	\\
				\rowcol	3	& Java ME	& Android	& javax.microedition.lcdui.Command	& 6 	& 3		& 3		& 3 	& 0		& 0.00	& 0	& 0		\\
				\rowpln	4	& Java ME	& Android	& javax.microedition.lcdui.Graphics	& 39	& 18	& 36	& 18	& 29	& 0.47	& 5	& 24 	\\
				\rowcol	5	& Java ME	& Android	& javax.microedition.lcdui.Font		& 16	& 3		& 15 	& 3		& 8		& 0.75	& 0	& 8		\\
				\rowmidlinecw
				\rowpln	6	& Java	& C\#	& java.io.File				& 54	& 15	& 37 	& 15	& 26	& 0.41	& 7		& 19	\\
				\rowcol	7	& Java	& C\#	& java.io.Reader			& 10	& 1		& 8		& 1		& 6		& 0.70	& 1		& 5	\\
				\rowpln	8	& Java	& C\#	& java.io.Writer			& 10	& 2		& 10 	& 2		& 10	& 0.80	& 1		& 9	\\
				
				\rowmidlinewc
				\rowcol	9	& Java	& C\#	& java.util.Calendar$^*$	& 47	& 0		& 11	& 0		& 5		& 0.24	& 0 	& 5	\\
				\rowpln	10	& Java	& C\#	& java.util.Iterator$^*$	& 3		& 0 	& 3 	& 0 	& 1		& 1.00	& 0		& 1	\\
				\rowcol	11	& Java	& C\#	& java.util.HashMap			& 17	& 5		& 9	 	& 5		& 5		& 0.24	& 1		& 4	\\
				\rowpln	12	& Java	& C\#	& java.util.ArrayList		& 28	& 6		& 22	& 6		& 15	& 0.58	& 4		& 11 \\
				
				\rowmidlinewc
				\rowcol	13	& Java	& C\#	& java.sql.Connection		& 52	& 1		& 28	& 1		& 13	& 0.52	& 1		& 12 \\
				\rowpln	14	& Java	& C\#	& java.sql.ResultSet		& 187	& 10	& 146	& 10	& 31	& 0.73	& 1		& 30 \\
				\rowcol	15	& Java	& C\#	& java.sql.Statement		& 42	& 1		& 21	& 1		& 5		& 0.48	& 1 	& 4	\\
				
				\bottomline
				\rowpln	Total&		& 		& 							& 549	& 73	& 382 	& 73 	& 171	& 0.57$^{**}$ 	& 22	& 149	\\
				\bottomline
				\rowpln \multicolumn{12}{r}{{$^*$=Previous approach reported a manually constructed class as mapping; $^{**}$=Average}} \\
				\rowpln \multicolumn{12}{r}{{\footnotesize Prev= previous approach; Previous approach for Java ME-Android mappings is Rosetta~\cite{Gokhale2013ICSE}; Previous approach for Java-C\# mappings is StaMiner~\cite{nguyen2014statistical}}} \\
				%----------------- END TABLE DATA ------------------------ 
		\end{tabular}
		\label{tab:RosettaComp}
	\end{center}
\end{table*}

Table~\ref{tab:RosettaComp} presents our evaluation results for answering RQ1. 
The columns `API' lists the name of source API under `Source' and target API under `Target'. 
The column `Type' lists the class or interface in source API
under consideration for finding mappings in target API.
The column `No. Methods' lists the number of methods in the class or interface under consideration.
We only consider the methods explicitly declared or overridden by a type and ignore the inherited methods.
The column `Relevant' lists the number of methods for which at least one relevant mapping is reported.
The sub-column `Prev.' reports relevancy numbers by previous approaches.
The previous approach for comparison of Java ME-Android mappings is Rosetta~\cite{Gokhale2013ICSE}.
The previous approach for comparison of Java-C\# mappings is StaMiner~\cite{nguyen2014statistical}.
The sub-column `\tool' reports relevancy numbers by \tool\ (at least one relevant method in top-ten results).
The column `Exact' lists the number of methods for which a exact mapping is found.
The sub-column `Prev.' reports exact numbers by previous approaches.
Since previous approaches do not make a distinction between exact and relevant, we report same values for both columns.
The sub-columns `\tool' reports exact numbers by \tool\ (at least one exact method mapping in top-ten results).
Column `$\Delta_{Cov}$' lists the ratio of increase in the number of methods for which a relevant mapping was found by \tool\ to the total number of methods in the type.


Our evaluation results indicate the \tool\ on an average finds relevant mappings for 57\% (Column `$\Delta_{Cov}$') more methods. 
For the Java ME-Android mappings \tool\ performs best for \CodeIn{Alert}
and \CodeIn{Font} classes from \CodeIn{javax.} \CodeIn{microedition.lcdui}
package in Java ME API with 75\% increase in number of methods
for which a relevant mapping was found in Android API.
For the Java-C\# mappings \tool\ performs best for \CodeIn{Iterator} interface from \CodeIn{java.util} package in Java API
finding a relevant method in C\# API for all the methods.
Previous approach StaMiner reports a manually constructed wrapper type as a mappings instead.
Furthermore, our results also indicate that \tool\ found on average exact mappings for 6.5 ($(171-73)/15$) more methods per type with a maximum of 21 additional exact mappings for a \CodeIn{java.sql.ResultSet} type as compared to previous approaches.
We next describe the cases where \tool\ did not find any relevant mapping.

A major cause for inadequate performance of \tool\ is 
lack of one-to-one mapping between methods in source and target API.
Often times functionality of a method in a source API 
is broken down into multiple functions in the target API or vice versa.
Although, \tool\ reports some of the relevant methods, exact mapping may involve a sequence of method calls in target API which is the limitation of \tool.
In the future, we plan to investigate techniques to deal with such cases.


Another cause of inadequate performance of \tool\ is 
inconsistent use of terminology across different APIs.
For instance, \tool\ did not find any additional relevant mapping for methods in \CodeIn{Command} class in Java ME API.
In Java ME API, `command' is used to refer the user interface construct `button'.
In Android API, `command' is used in more conventional sense of the term.
This inconsistent use of terminology causes \tool\ to return irrelevant results. 
When we manually replaced the term `command' with `button' in the generated queries,
we observed a relevant method appeared in top ten results for every method \CodeIn{Command} class in Java ME API.
However, we refrain from including such modifications to stay true to \tool\ approach for evaluation.
In the future, we plan to investigate techniques to automatically suggest alternate keywords.


\subsubsection{RQ2: Quality of discovered mappings}

To answer RQ2, we compared the exact mappings discovered by \tool\
with the mappings discovered by previous approaches.
In Table~\ref{tab:RosettaComp} the previous approach for comparison of the Java ME-Android mappings is Rosetta~\cite{Gokhale2013ICSE} and the
previous approach for comparison of Java-C\# mappings is StaMiner~\cite{nguyen2014statistical}.
Our results show that out of 171 discovered exact mappings only 22 are in common with previous approaches.
We next discuss some of the implications of the results.

Before carrying out this evaluation, we had hoped that
the mappings discovered by \tool\ would significantly
overlap with the mappings discovered by the Rosetta and StaMiner,
as these approaches infer mappings from actual source code.
Thus, these mappings can be considered as the representative of how developers are actually migrating software. 
However, the results suggest a low overlap.
We manually investigated the possible \tool\ specific implications of the observed mismatch.


The results (matches found) comprise of methods from different classes in the target API,
reflecting that often there are multiple ways to solve a problem, or to implement a feature using an API. 
Further, choice of using multiple APIs gives a developer the flexibility to use different approaches when porting an application from one platform to another. 

When more than one match is found for a given source method, the results in \tool\ are ranked according to the similarity score, with the more relevant (or exact matches) ranked higher. 
The ranked set of results helps the developer use the best suited or most appropriate target method in their implementation. This approach is different from earlier approaches \cite{Gokhale2013ICSE,nguyen2014statistical} that focused on only exact matches between different classes based on the number of methods within them that are similar. 


We also contacted the authors of Rosetta approach~\cite{Gokhale2013ICSE} to report the difference in the mappings.
Specifically, we inquired that Rosetta does not report any method form \CodeIn{AlertDialog} in the Android API as a possible mapping for
\CodeIn{Alert.setString} method in Java ME API.
Rosetta reports method sequence \CodeIn{Paint.setAlpha} \CodeIn{CompundButton.} \CodeIn{setChecked} as one of the likely mappings.
In contrast, \tool\ discovers \CodeIn{AlertDialog.setTitle} method in Android API as a likely mapping.

The lead author of Rosetta approach responded that they restricted the output of
Rosetta to sequences of length up to 2 when inferring mappings (i.e., {A$\rightarrow$p;q}, or {A$\rightarrow$p}). Furthermore, authors count a reported method sequence as a valid mapping
if the reported sequence implements some of the functionality of a source method on the target platform.
With regards to our query, Rosetta authors observed that in many of the traces, setting a \CodeIn{string} first involves setting the attributes of the \CodeIn{Paint} (with is used to draw the text), followed by a call to \CodeIn{setText} method, which led them to believe that the sequence \CodeIn{Paint.setAlpha} \CodeIn{CompundButton.} \CodeIn{setChecked} could be a likely mapping, if at least in part.
Although, author did confirm that technically \CodeIn{AlertDialog.setTitle} method in Android API as a likely mapping.

The exchange with Rosetta approach's lead author points out that
the mappings discovered by \tool\ generally point to
the closest aggregate API in contrast to the individual smaller API calls
that achieve same functionality.
Furthermore, exchange also demonstrates the reliance of the
code analysis approaches on the quality of the code under analysis.
In contrast, \tool\ relies on the quality of the API method descriptions.


%\subsubsection{RQ3: Effectiveness in free form queries}
%
%RQ3 demonstrates the effectiveness of text-mining in general for API related information retrieval tasks.
%In particular, we show the effectiveness of creating vector space representation of
%API method descriptions by using free form queries.
%For evaluating RQ3, we use the same queries as used by Chatterjee et al.'s~\cite{chatterjee2009sniff} in evaluation of their approach Sniff.
%Sniff is targeted towards searching for relavant method snippets from source-code repositories. Sniff first annotates the source code with API document descriptions and uses the hybrid representation of source code to get relevant code fragments. 
%Since \tool\ does not report method sequences, we consider a method reported by \tool\ as relevant iff:
%
%\begin{enumerate}
%	\item the method is in top-10 results.
%	\item the method is one of the methods in the code fragment reported by Sniff.
%\end{enumerate}
%
%Table~\ref{tab:SNIFFComp} shows the effectiveness of \tool\ using free form queries.
%The column `Query' lists the query used in evaluation.
%The column `Sniff' lists the ranking of relevant code fragment by Sniff as reported in their work.
%The column `\tool' lists the ranking of first relevant method returned by \tool.
%
%
%Our evaluation results show that in \tool\ returns the relevant method in top 10 results (except one) demonstrating effectiveness of \tool\ with free form queries. For the query `read a line of text from a file' all the reported result methods did support the functionality of reading lines from file, however were not reported as relevant code fragment by Sniff.
%
%\begin{table}
%	\begin{center}	
%		\caption{Comparison with Sniff}
%		\begin{tabular}{lrr}
%			\topline
%			\headcol Query	& \multicolumn{2}{c}{Method Rank}\\
%			\headcol 		& {\small Sniff}	& {\small \tool} \\
%			\midline 
%			
%			\rowcol get active editor window from eclipse workbench	& 1 & 1\\
%			\rowpln parse a java source and create ast & 1 & 2\\
%			\rowcol connect to a database using jdbc & 1 & 6\\
%			\rowpln display directory dialog from viewer in eclipse & 1 & 1\\
%			\rowcol read a line of text from a file & 1 & -\\
%			\rowpln return an audio clip from url & 1 & 1\\
%			\rowcol execute SQL query & 2 & 3\\
%			\rowpln current selection from eclipse workbench & 1 & 1\\ 
%			\bottomline
%			%----------------- END TABLE DATA ------------------------ 
%			\rowpln \multicolumn{3}{r}{{\small `-'=No Match in top-10 results.}}\\ 
%			\bottomline
%		\end{tabular}
%		\label{tab:SNIFFComp}
%	\end{center}
%\end{table}

