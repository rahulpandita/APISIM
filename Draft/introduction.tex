\section{Introduction}
\label{sec:introduction}


Software is ubiquitous and lately people are increasingly interacting with applications
that run on variety of software platforms on their consumer devices on daily basis.
For instance, a typical user may view emails on his/her android or iOS or windowsPhone handheld
device such as a mobile phone or a tablet while on the go. 
The user then can switch to a desktop environment such as Windows or Linux or OSX 
in an office like setting through either a web browser or 
a standalone application such as Microsoft Outlook or Mozilla Thunderbird.


To retain users across different platforms,
developers are increasingly releasing different versions of their applications.
For example, a mobile software developers often release their applications
for all the popular mobile platforms such as Android, iOS, and Windows.
In context of desktop software, many well-known projects such as JUnit,
Hibernate provide multiple versions in different languages,
in an attempt to attract developer community to use these
libraries across different languages.


However, manually migrating a software from one platform (/language) to another
can be time consuming and may be error prone.
Existing language migration tools such as Java2CSTranslator~\cite{java2cstranslator}
partially alleviate the problem, as they require a programmer to manually describe how Application Programming Interfaces (APIs) of a source platform(/language) maps to API's of the target platform (/language). 
Given a typical platform (/language) expose a large number of API methods for developers to reuse, manually writing these mappings is prohibitively resource intensive and may result in manual error.


\textit{The goal of this research is to support software developers 
in migrating an application from one API to another
by automatically inferring method mappings across API using text mining
on the natural language API method descriptions}


In the literature, there exist approaches that address the problem of finding mapping between  software API's using static~\cite{Zhong2010ICSE} and dynamic~\cite{Gokhale2013ICSE} analysis. 
Recently Nguyen et al.~\cite{nguyen2014statistical} also proposed to apply statistical language translation techniques to achieve language mapping by mining large open source software repositories.  
However, aforementioned approaches rely on the presence of manually ported (or at least functionally similar) software across source and target API's.
Furthermore, accuracy of static analysis and mining based approaches~\cite{Zhong2010ICSE,nguyen2014statistical} is dependent upon the quality of the code under analysis.
Likewise, accuracy of dynamic approaches like \cite{Gokhale2013ICSE} is function of the quality and completeness of test case to execute the comprehensive API behavior dynamically. 


To address the shortcomings of existing program-analysis based approaches,
we propose to use the natural language API method descriptions
to infer method mappings across APIs.
Our key intuition is:
\textit{``since the documentation is targeted towards developers,
	there may be an overlap in the language used to describe similar concepts that can be leveraged.''}
In general, API documentation provides developers with useful information
about class/interface hierarchies within the software.
Additionally, API documents also provides information about
how to use a particular method within a class by means of method descriptions.
Method descriptions typically describe specifications in terms of
the expectations of the method arguments and functionality of method in general.


This paper presents  \tool : An approach that leverages the natural language method descriptions to find the likely mapping.
In particular, \tool\ proposes to create a vector space model~\cite{singhal2001modern,manning2008introduction} of the target API method descriptions. 
\tool\ then queries the vector space model of target API using  
automatically generated queries from the source API method descriptions.
\tool\ automates the query generation in source API using the concepts from text mining such as emphasizing (or de-emphasizing) certain keywords over others and querying multiple facets (such as class description, package names etc...).


We pose the following research question :
\textit{How accurately can the similarity in the language of API method descriptions
	be leveraged to infer API Mappings?}
To answer our question, we apply \tool\ to infer API mappings across:
1) \CodeIn{Java} and \CodeIn{C\# API}; 2) \CodeIn{J2ME} and \CodeIn{Android} API.
We next compare the inferred mappings with state-of-the-art code mining based approaches~\cite{nguyen2014statistical,Gokhale2013ICSE}.
Our results indicate that \tool\ is effective in inferring API mappings with the more than XX accuracy.

In summary, \tool\ leverages natural language description of API's to infer likely mapping thus facilitating cross API migration of applications. As our approach analyzes API documents in natural language, it can be reused independent of the programming language of the library. Our paper makes the following major contributions:
\begin{itemize}
	\item A technique that effectively infers mapping across source and target API.
	\item A prototype implementation of our approach based on extending the Apache Lucene~\cite{lucene}. An open source implementation of our prototype can be found at our website~\footnote{\url{http://xx.com/}}. 
	\item An evaluation of our approach on J2ME to Android API and Java to C\# API
\end{itemize}


The rest of the paper is organized as follows.
Section~\ref{sec:example} presents an real world examples that motivate our approach. Section~\ref{sec:related} discusses related work.
Section~\ref{sec:background} presents the  background on Text Mining.
Section~\ref{sec:approach} presents our approach.
Section~\ref{sec:evaluation} presents evaluation of our approach.
Section~\ref{sec:discussion} presents a brief discussion and future work.
Finally, Section~\ref{sec:conclusion} concludes.


